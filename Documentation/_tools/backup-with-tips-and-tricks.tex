\documentclass[12pt,a4paper]{article}
%\usepackage[german]{babel} %Für die indirekte Angabe von Umlauten. Es müssen dann Umlaute wie folgt im Code angegeben werden: "a "o "u "s.

\usepackage[utf8]{inputenc}
%dieses Paket ermöglicht uns, Umlaute im Text als solche eingeben zu können (Windows/Linux)

\usepackage{amsmath, amsthm, amssymb}
\usepackage{enumerate}
\usepackage{graphicx}
\usepackage{lscape}
\usepackage{setspace}
\onehalfspacing
\usepackage{wrapfig}
\usepackage{hyperref}% für die Einbettung von Hyperlinks
\usepackage{multirow}
\usepackage[round]{natbib}
\bibliographystyle{apalike}

\newtheorem{definition}{Definition}[section]
\newtheorem{interview}{Interview}[section]
\newtheorem{satz}{Satz}[section]
\newtheorem{beispiel}{Beispiel}[section]
\newtheorem{bemerkung}{Bemerkung}[section]
\newtheorem{literaturverzeichnis}{Literaturverzeichnis}[section]

\usepackage[top=25mm, bottom=20mm, vmargin=25mm, includehead]{geometry} % Document Margins
\setlength{\topmargin}{0cm}
\setlength{\parindent}{5mm}
\setlength{\parskip}{2mm}
\setlength{\evensidemargin}{0mm}
\setlength{\oddsidemargin}{0cm}
%\pagestyle{headings}

\providecommand{\tightlist}{%
  \setlength{\itemsep}{0pt}\setlength{\parskip}{0pt}}

\begin{document}
\thispagestyle{empty}
\vspace*{-3cm}
\begin{center}
\large \textsc{}
\vspace{0.5cm}
%\hrule
\vspace{5.5cm}
{\large}\\
\vspace{1cm}
{\Large \bf
Automatically Estimating Task Durations}\\
\vspace*{1cm}
{\large using arithmetic and artificial intelligence}
\end{center}
\vspace*{14cm}


\hspace*{\fill} from Stefan Hoffmann and others, \today

\newpage
\pagenumbering{Roman}
\tableofcontents

%\newpage
%\addcontentsline{toc}{section}{tables} %wenn es im Inhaltsverzeichnis erscheinen soll - sonst auskommentieren mit "%"
%\listoftables

%\newpage
%\addcontentsline{toc}{section}{Abbildungsverzeichnis} %wenn es im Inhaltsverzeichnis erscheinen soll - sonst auskommentieren mit "%"
%\listoffigures

\newpage
\pagenumbering{arabic}
%Und nun kommen wir zur Arbeit und fangen an die Seiten mit Arabischen Zahlen zu zählen
\hypertarget{foreword}{%
\section{Foreword}\label{foreword}}

I want to make it short. Guessing task durations is a mess. You can put a lot of effort in making time estimates. Still they will be not good. And while you strife to make perfect estimates you will find yourself investing more and more time - possibly a lot more than the task is worth.

Unfortunately we cannot fully discard them, since guessing

With automation you can get more benefits and less pain:

\begin{itemize}
\tightlist
\item
  Since the algorithms are based on actual data, they are not "just wild guesses". They can watch a particular pipeline and become better and better according to the actual output.
\item
  Algorithms are systematic. They are well documented in terms of how they guess. This way they do not trigger the more human problems when guessing time estimates: The boss will half this guess, so I will better double it... The boss sees your estimate and cuts it down to 15 Minutes because "what could be the problem?".
\item
  Because of works like this e-book and scientific methods available to us algorithms can be validated. This means we can become better and better.
\item
  It is more easy to communicate, that it is a guess, not a fixed date. The algorithms will give you a number. It does not know what it is actually guessing. 
\item
  Business people can have their time estimates in nearly real time.
\item
  The developers of those business peoples can do other things in the time they would spend guessing. For example they can code. This means they are more productive.
\item
  The developers are more happy because they do not loose time to guessing and they have no need to fight all the people off that ask for an estimate just to ridicule it and then punish developers for not reaching the deadline that they faked. 
\end{itemize}

Let us see, if the idea is working or not. Maybe not. But, since the only thing we have to loose is bad estimates and the only thing we can gain is bad estimates that do not make effort - what the heck.

\hypertarget{accompagning-material}{%
\section{Accompagning Material}\label{accompagning-material}}

With this booklet I release:

\hypertarget{estimateps}{%
\subsection{EstimatePS}\label{estimateps}}

EstimatePS is a powershell commandlet to estimate tasks from the command
line, implemented but not limited to Powershell Core. The code is part
of this repository. But it is also available through the Powershell
Gallery.

\begin{verbatim}
Install-Module EstimatePS
\end{verbatim}

May it be of service to those with powershell and the need for time estimates.

\newpage{}

\hypertarget{quality-assurance}{%
\section{Quality Assurance}\label{quality-assurance}}

Now if we create algorithms we need to say how good they are. For that
we are using the following methods.

\hypertarget{the-graphical-approach}{%
\subsection{The graphical approach}\label{the-graphical-approach}}

One easy way would be to draw a graph which shows the predictions and
the real values for time spent, while each task is understood as a
category of its own.

We sort that graph by actual duration so we should see the distribution
of durations and around that a hopping range of dots that describes what
the algorithm tells us.

A better algorithm should be closer to the real data. Any algorithm
should never match perfectly, as then we would have a 1:1 mapping. And
that is an over-fit for sure.

\hypertarget{mean-squared-error}{%
\subsection{Mean squared error}\label{mean-squared-error}}

The mean squared error is a common approach to calculate a value for the
quality of an algorithmi. It gets bigger with every estimate we did
wrong.

The formula can be described as:

For every value you predict:

\begin{itemize}
\tightlist
\item
  Calculate the difference between the predicted value and the real
  value
\item
  Sum the squares of each difference
\item
  Divide all the sum by the count of the data entries you check
\end{itemize}

Now, since we want to prevent overfitting we need to prevent
underfitting as well. Since we are talking about seconds and most of the
recorded tasks have a duration in the range of up to 50000 seconds that
means that most tasks are completed in about 13,89 hours. So what about
an error margin of about 5 hours. Which means just as something to think
of, we want the squared error to not exceed squared(5x60x60) =
324.000.000 .

\hypertarget{above-and-below}{%
\subsection{Above and below}\label{above-and-below}}

As a third criteria we have the idea that estimations might even each
other out. In a prefect scenario this would mean that 50\% of the
estimations are too high while the other 50\% are too low. To find out
how good we match we add 1 to a variable for every estimation we find
above the real value and then divide it by the number of tasks. The
result should be .5 when hitting the target.

\hypertarget{the-data}{%
\subsection{The Data}\label{the-data}}

For training and estimating the quality of algorithms I use herein a
dataset that consists of all the tasks that we at the software
development shop at my employer recorded during the time between july
and december 2020 and january 2021 to august 2021. That means there are
two datasets available.

Unfortunately - since this is confidential information - I cannot
publish it alongside this material. But the errors and images can be
shared at it can advance the algorithms - and it is everything that I
have right now. So that will do.

I'll call them: swe2020 and swe2021.

Now let us get a glance at the data as well the first algorithm.


\newpage
\section{Hauptteil}\label{body}
\markright{Hauptteil}
mmmmm ipsum dolor sit amet, consetetur sadipscing elitr, sed diam nonumy eirmod tempor invidunt ut labore et dolore magna aliquyam erat, sed diam voluptua. At vero eos et accusam et justo duo dolores et ea rebum. Stet clita kasd gubergren.

\subsection{Richtig Zitieren}
Richtig und einfach zitieren in \LaTeX\ setzt eine gut eingepflegte  .bib-Datei voraus.
\subsubsection {Literaturbibliothek: .bib-Datei }
In \LaTeX\ wird der Literaturnachweis mithilfe einer Literaturbibliothek gelöst. Man erstellt hierzu eine \textbf{\emph{.bib-Datei}}, die f"ur die einzelnen Werke die bibliografischen Daten (Autor, Titel, Erscheinungsjahr, Verlag, etc.) enthält. Jedes Werk bekommt dann eine eindeutige Bezeichnung. Bei einem Zitat wird dann die entsprechende Bezeichnung mittels eines \textbf{\emph{cite-Befehls}} angegeben. Um das Literaturverzeichnis anzeigen zu lassen, fügt man an der gewünschten Stelle im Dokument den \textbf{\emph{tableofcontects-Befehl}} ein.

\subsection{Auflistungen in \LaTeX}
Für Auflistungen in \LaTeX\ verwendet man verschidene Umgebungen. Mit den folgenden Umgebungen kann man Auflistungen mit bis zu vier Ebenen leicht erstellen.
\subsubsection{Stichpunkte mit \textbf{\emph{itemize}}}
{\setstretch{1.0}


\begin{itemize}%Die itemize Umgebung ermöglicht die unnummerierte Auflistung in Latex.
\item Stichpunkt X
\item Stichpunkt Y
\end{itemize}
\subsubsection{Nummerierte Auflistung mit \textbf{\emph{enumerate}}}
\begin{enumerate}
    \item Zeilenabstände
    \item können
    \begin{enumerate}
        \item temporär
        \begin{enumerate}
            \item reduziert
            \begin{enumerate}
                \item werden.
            \end{enumerate}
        \end{enumerate}
    \end{enumerate}
\end{enumerate}
}
\newpage % 

\subsection{Darstellung Mathematischer Ausdrücke}\label{math}
\markboth{Die Funktion}{Die Funktion}
\begin{equation}\label{AuszahlungCall}
Call_e[S,K,T,T]=\left[S(T)-K\right]^+
\end{equation}
Die Gleichung \eqref{AuszahlungCall} bezeichnet die Auszahlung der Call Option zum Zeitpunkt $T$.
\begin{eqnarray*}
Call_e[S,K,t,T]&=&S(t)\cdot N(d_1)-K\cdot e^{-(T-r)r}\cdot N(d_2),\\
d_{1/2}&=&\frac{\ln\left(\frac{S_t}{K\cdot e^{-(T-t)r}}\right)\pm\frac{1}{2}\sigma^2(T-t)}{\sigma\sqrt{T-t}}
\end{eqnarray*}
\begin{align}
\frac{\partial Call_e[S,K,t,T]}{\partial S}par&=N(d_1)+S\cdot\frac{\partial N(d_1)}{\partial S}-K\cdot e^{-r\cdot(T-t)}\cdot\frac{\partial N(d_2)}{\partial S}&t<T\\
\frac{\partial Call_e[S,K,t,T]}{\partial S}&=N(d_1)&t<T
\end{align}
$$Call_e[S_t,K,t,T]=S_t\cdot \underbrace{N(d_1)}_{\text{Hedgeratio}}-\underbrace{K\cdot e^{-r(T-t)}\cdot N(d_2)}_{Kassa-Hedge}$$

Hedge-Ratio: $\frac{\partial \text{Put}}{\partial S}=-N(-d_1)\leq 0$

\subsection{Tabellarische Darstellung}\label{numeric}
\markboth{Tabellarische Darstellung}{Tabellarische Darstellung}

Die Beispielstabelle \ref{Tbsp} dient der Darstellung der \textbf{\emph{table-Umgebung}} in \LaTeX.\\
Ausführliche Erklärungen und Beispiele finden Sie unter folgendem Link:\\
\href{https://www.overleaf.com/learn/latex/tables}{\textbf{\emph {Overleaf Guide to Tables.}}}
\begin{table}[h!]
\centering
\begin{tabular}{||c c c c||} 
 \hline
 Col1 & Col2 & Col2 & Col3 \\ [0.5ex] 
 \hline\hline
 1 & 6 & 87837 & 787 \\ 
 2 & 7 & 78 & 5415 \\
 3 & 545 & 778 & 7507 \\
 4 & 545 & 18744 & 7560 \\
 5 & 88 & 788 & 6344 \\ [1ex] 
 \hline
\end{tabular}
\caption{Beispieltabelle}
\label{Tbsp}
\end{table}
\newpage
\subsection{Bilder und Diagramme}
%\markboth{bilder}{Bilder}
%\begin{figure}[h]
%\includegraphics[width=\textwidth]{FS-Vorlage/Images/FS-Building.jpg}
%\caption{FS-Campus} \centering\cite{fs}
%\end{figure}
in \LaTeX\ kann man mithilfe des {\textbf{\emph {graphix-Pakets}}} Bilder und Diagramme jedem Dokument hinzufügen. Eine ausführliche Erklärung und Beispiele finden Sie unter folgendem Link:\\
\href{https://de.overleaf.com/learn/latex/Inserting_Images}{\textbf{\emph{Overleaf Guide to Inserting Images}}}

\subsection{Hyperlinks}
\markboth{Hyperlinks}{Hyperlinks}
\LaTeX\ ermöglicht uns, im Text mithilfe des \textbf{\emph{href-Befehls}}  Hyperlinks einzubetten:\newline
\href{https://www.overleaf.com/learn/latex/Hyperlinks}{\textbf{\emph{Overleaf Guide to Hyperlinks}}}.\newline
Alternativ kann man den {\textbf{\emph{url-Befehel}}} einsetzen: \url{https://www.frankfurt-school.de}

\begin{appendix}
\section{Herleitung}
\markboth{Appendix}{Appendix}
%Siehe \citet{sandmann}.

\end{appendix}




\newpage
\markboth{}{}
%\begin{thebibliography}{99}
%\bibitem{sandmann} Sandmann, K.; (2010): {\it Einführung in die Stochastik der Finanzmärkte}. Springer-Verlag, Berlin.
%\end{thebibliography}

\newpage
\addcontentsline{toc}{section}{Literaturverzeichnis} %wenn es im Inhaltsverzeichnis erscheinen soll - sonst auskommentieren mit "%"
\renewcommand{\refname}{Literaturverzeichnis}
\bibliography{FS-Literature}

\newpage
\thispagestyle{empty}
\markboth{}{}
  \normalsize
\begin{center}
\huge{\bf Eigenständigkeitserklärung}\\[40mm]
\end{center}
\large
Ich versichere, dass die vorstehende Arbeit von mir selbstst"andig ohne unerlaubte fremde Hilfe und ohne Benutzung anderer als der angegebenen Hilfsmittel angefertigt wurde, und dass ich alle Stellen, die w"ortlich oder sinngem"a"s aus ver"offentlichten oder unver"offentlichten Schriften entnommen sind, als solche gekennzeichnet habe.\\[50mm]
Frankfurt, den \today

\newpage



\end{document}

