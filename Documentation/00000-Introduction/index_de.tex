\hypertarget{foreword}{%
\section{Vorwort}\label{foreword}}

Um es kurz zu machen: 

Aufgabendauern schätzen ist im Grunde eine Katastrophe. 

Neben vielen anderen Effekten, wie Arbeitszeitverlust durch Schätzaufwand: Wenn ein Mensch schätzt, entsteht daraus ein emotionales Durcheinander. Der Chef schätzt die Aufgabe zu gering? Man selbst zu hoch? Wie kann der nur! Das ist nicht gut, denn eigentlich wollen wir alle das gleich: großartige Software.

Leider werden wir gleichzeitig Schätzungen nicht vollständig los, weil wirtschaftliches Handeln nun mal eine entsprechende Betrachtungsweise vor Beginn einer Arbeit voraussetzt. 

Algorithmen sind neutrale Dritte, die sich an den Daten orientieren, welche wir Ihnen geben. Vorraussichtlich sind das echte Daten aus echten Messungen. Sie sind nachvollziehbar, prüfbar und können Schrittweise verbessert werden.

Dazu kosten uns diese Zeitschätzungen weniger Zeit. (Humor am Rande: Jedenfalls allen außer mir, denn diese Arbeit hat schon mehr Zeit in Anspruch genommen, als ich je gedacht hätte.)

Dieses E-Book basiert auf den Messungen und Experimenten, die wir in der Abteilung bei meinem Arbeitgeber durchführen konnten. Zugegeben haben wir keine besonders große Datenbasis oder viele Mitarbeiter. Vor kurzem waren wir noch 3, jetzt sind wir 2 Entwickler. Vielleicht erlaubt uns diese Größe aber auch noch, qualitative Betrachtungsweisen einfacher anstellen zu können. 

Wir zeichnen die Arbeitszeit, die wir mit Aufgaben verbringen auf. Und das nun schon eine ganze Weile, was zwei Datensets mit jeweils etwa 1000 Aufgaben erschaffen hat. 

Das E-Book ist aber so entwickelt, dass Sie sich das Buch und den zugehörigen Code auf github forken können und die Algorithmen und ihre Prüfungen über ihre eigenen Datenbestände laufen lassen können. Schließlich wollen Sie ja sicher Ihre Aufgaben schätzen, und nicht unsere.

Ich wünsche Ihnen viel Spaß und Erfolg,\\
Stefan Hoffmann\\
Berlin, den 23.10.2021

