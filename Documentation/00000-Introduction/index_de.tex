\hypertarget{foreword}{%
\section{Foreword}\label{foreword}}

I want to make it short. Guessing task durations is a mess. You can put a lot of effort in making time estimates. Still they will be not good. And while you strife to make perfect estimates you will find yourself investing more and more time - possibly a lot more than the task is worth.

Unfortunately we cannot fully discard them, since guessing

With automation you can get more benefits and less pain:

\begin{itemize}
\tightlist
\item
  Since the algorithms are based on actual data, they are not "just wild guesses". They can watch a particular pipeline and become better and better according to the actual output.
\item
  Algorithms are systematic. They are well documented in terms of how they guess. This way they do not trigger the more human problems when guessing time estimates: The boss will half this guess, so I will better double it... The boss sees your estimate and cuts it down to 15 Minutes because "what could be the problem?".
\item
  Because of works like this e-book and scientific methods available to us algorithms can be validated. This means we can become better and better.
\item
  It is more easy to communicate, that it is a guess, not a fixed date. The algorithms will give you a number. It does not know what it is actually guessing. 
\item
  Business people can have their time estimates in nearly real time.
\item
  The developers of those business peoples can do other things in the time they would spend guessing. For example they can code. This means they are more productive.
\item
  The developers are more happy because they do not loose time to guessing and they have no need to fight all the people off that ask for an estimate just to ridicule it and then punish developers for not reaching the deadline that they faked. 
\end{itemize}

Let us see, if the idea is working or not. Maybe not. But, since the only thing we have to loose is bad estimates and the only thing we can gain is bad estimates that do not make effort - what the heck.
