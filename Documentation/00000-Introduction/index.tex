\hypertarget{foreword}{%
\section{Foreword}\label{foreword}}

To put it all in a nutshell: Estimating task durations is a mess. You can invest a lot of effort and still they end up mediocre in quality.
And while you keep pushing for perfection you find yourself investing more and more time ending up in the situation that the estimation finally takes longer than the task itself.

Unfortunately we cannot get rid of them, because we need to do stuff economically. This kinda involves planning.

But with automation we can get more benefits and less pain:

\begin{itemize}
\tightlist
\item
    When letting algorithms do the hard work they do their estimating (hopefully) based on historically collected data. So what they do is a lot better than the more or less "random stuff" we as humans tend to do. Those algorithms additionally can continuiously watch a pipeline (like our software development department) and keep improving by learning from every new task we complete thus adopting to changes in the way we work fastly.
\item
    Algorithms are systematic. They are well documented in terms of how they guess. This way they do not trigger the more human problems when guessing time estimates, e.g.: "The boss will half this guess, so I will better double it..." or "The boss sees your estimate and cuts it down to 15 Minutes because "what could be the problem?". (Ok, to be perfectly honest, those algorithms do not per se keep your boss from maybe acting this way. But you gain the fun in asking why they actually think that their guess is any better then this or that method.)
\item
  Because of works like this e-book and scientific methods available to us algorithms can be validated. This means we can become better and better.
\item
  It is more easy to communicate, that it is a guess, not a fixed date. The algorithms will give you a number. While a lot indicates that those algorithms should perform better than a human it is still clear, that they are just algorithms. While what they do makes kinda sense they for sure do not know what it is they are actually guessing. 
\item
    Business people can have their time estimates in nearly real time. 

    (Agreed they will probably start playing around with their task descriptions sooner or later until they have the feeling that the estimate is more to their liking. 
    But I always think of two things there: 
    First, all they do is cut themselfs. If they like it? 
    Second, maybe the algorithm tells the truth? Its descisions are based on historic data and logic. So maybe the change in the task description is actually really describing a cheaper solution to the same problem?)
\item
    A very important point from my perspective is that the developers of algorithm-users are more productive.
        I'll use dark humor to explain: 
        When they keep us estimating projects for the 40 hours of the week how on earth can business people be sometimes shocked by the circumstance that at the end of the week those tasks are not completed yet? 
\item
    Developers love coding. Coding makes us happier. We want to create new stuff. When we no longer have to fight with the people that want estimates (sometimes only to ridicule them and then punish us for deadlines that only existed in a world of miracles and unicorns) we can do just that. Code. Get those eggs from that basket everybody is talking about.
        Win-win.
\end{itemize}

Let us see, if the idea is working or not. Maybe not. But, since the only thing we have to loose is bad estimates and the only thing we can gain is bad estimates that do not make effort - what the heck.
