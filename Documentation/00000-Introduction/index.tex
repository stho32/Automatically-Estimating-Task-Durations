\hypertarget{foreword}{%
\section{Foreword}\label{foreword}}

I want to make it short. Guessing task durations is a mess. You can put
a lot of effort in making time estimates. Still they will be not good.
And while you strife to make perfect estimates you will find yourself
investing more and more time - possibly a lot more than the task is
worth.

More than that: estimating tasks hurts. Until noone wants to estimate
task durations anymore. It is a pain for developers and business people 
alike.

With automation you can get more benefits and less pain:

\begin{itemize}
\tightlist
\item
  It is clear, that it is just an estimate. Even the most stubborn people have to admit it, because an algorithm just tells you an objective truth and it doesn't really know what it is talking about.
\item
  Algorithms are systematic. Although they do not know what they are actually guessing they are well documented in terms of how they guess. This way they do not trigger the more human problems when guessing time estimates: My boss will half this guess, so I will better double it... Your boss sees your estimate and cuts it down to 15 Minutes because "what could be the problem?": Not a problem anymore.
\item
    And computers can learn from big amounts of data. People can only learn so much, until they are stuck.
\item
  Business people can have their time estimates in real time.
\item
  The developers of those business peoples can do other things in the time they would otherwise need to make "better guesses" (which are also wrong). Most of the time that will mean they code. At since they will probably do that the productivity goes up for the same money.
\item
  The developers are more happy because they do not loose time to guessing and they have no need to fight all the people off that know better.
\end{itemize}

Let us see, if the idea is working or not. Maybe not. But, since the
only thing we have to loose is bad estimates and the only thing we can
gain is bad estimates that do not make effort - what the heck.
