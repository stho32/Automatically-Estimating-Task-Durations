\newpage{}

\hypertarget{a003---10-categories}{%
\section{A003 - 10 Kategorien}\label{a003---10-categories}}

\hypertarget{description}{%
\subsection{Description}}

Um besser als A001 zu werden könnten wir Kategorisierung anwenden.
Dazu teilen wir die Aufgaben in 10 Gruppen mit ähnlichen Eigenschaften auf.
Wenn man bei A001 betrachtet, sieht man dass es (zumindest in unseren Datenmengen) viele Aufgaben mit weniger als 30 Minuten und 1 Stunde Dauer gibt. D.h. wenn man einigermaßen sicher sagen könnte, dass eine Aufgabe zu einer entsprechenden Gruppe gehört, dann könnte man Streuungen verhindern, die A001 eben produziert.

Aber wie machen wir das?

\subsubsection{A003.1}

Wir teilen die Aufgaben entlang der Dauer auf
\begin{enumerate}
\tightlist
\item kleiner als 30 Minuten,
\item mehr als 30 Minuten und kleiner als 1 Stunde,
\item mehr als 1 Stunde und kleiner als 2 Stunden
\item mehr als 2 Stunden und kleiner als 3 Stunden
\item mehr als 3 Stunden und kleiner als 4 Stunden
\item mehr als 4 Stunden und kleiner als 5 Stunden
\item mehr als 5 Stunden und kleiner als 6 Stunden
\item mehr als 6 Stunden und kleiner als 7 Stunden
\item mehr als 7 Stunden und kleiner als 8 Stunden
\item mehr als 8 Stunden
\end{enumerate}

Unser Algorithmus versucht wichtige Worte zu identifizieren, Worte die pro Kategorie eindeutig sind. Und zwar so:
\begin{enumerate}
\tightlist
\item Sammle alle Worte zusammen, die in einer Kategorie vorkommen
\item Für jede einzelne Kategorie:
    \begin{enumerate}
        \tightlist
        \item lösche alle Worte aus der eigenen Liste, die in irgendeiner der anderen Kategorien vorkommen
    \end{enumerate}
\end{enumerate}