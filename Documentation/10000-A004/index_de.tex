\newpage{}

\section{A004 - reducing dispersion by assigning a concrete value per
word and learning from
it}

A001 sammelt so viele unterschiedliche Werte pro Wort, wie er bekommen kann. 
Wenn er dann schätzt, nutzt er die gesammelten Werte zufällig (natürlich etwas weniger zufällig, wenn man den Fakt betrachtet, dass er 100 Zufallswerte zieht, sie nach der Größe sortiert und dann nur die jeweiligen Werte nutzt, die eine bestimmte Wahrscheinlichkeit der Zielerreichung aussagen sollten...).
Das behindert etwas die Möglichkeit A001 etwas lernen zu lassen.

A004 soll dieses Manko ausgleichen, in dem für jedes Wort nur ein Wert hinterlegt wird. Das macht es einfacher auszuprobieren, ob man den Gesamtfehler des Algorithmus nicht ausgleichen kann, in dem man den Wert etwas erhöht oder vermindert.

\subsection{Lernen - Initialisierung}

Anfangs lernen wir auf die gleiche Weise wie A001. 
Der einzige Unterschied ist, dass wir die Einzelwerte nach dem ersten Lernen alle zu Durchschnitten zusammenziehen. 

\subsection{Lernen - Anpassung}

\begin{enumerate}
        \tightlist
        \item Ausgehend vom Hauptmodell erstellen wir 10 zufällige Mutationen, in dem wir pro Wort 1/10tel des Wertes abziehen oder aufaddieren.
        \item Wir rechnen für jedes Teilmodell den durchschnittlichen quadratischen Fehler aus. Das Modell mit dem geringsten Fehler wird das neue Hauptmodell.
        \item Wiederhole n mal vom Anfang an.
\end{enumerate}

\subsection{Schätzen}

\begin{enumerate}
        \tightlist
        \item Splitte den zu schätzenden Text in Worte.
        \item Für jedes Wort wird der im Modell gespeicherte Wert eingesetzt.
        \item Die Summe der Einzelwerte ist die Schätzung der neuen Aufgabe.
\end{enumerate}


