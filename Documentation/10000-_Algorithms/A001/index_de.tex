\newpage{}

\subsection{A001 - Dauer pro Wort}

Wir beginnen mit einem sehr sehr einfachen Algorithmus, dem tatsächlich ersten, 
der mir in den Sinn kam.

Sagen wir, wir haben da eine Aufgabenbeschreibung. Vielleicht "Kaufe ein Buch
und lege es im Regal ab.". Und diese Aufgabe dauert eine bestimmte Zeit.

\begin{verbatim}
Dauer = "Kaufe ein Buch und lege es im Regal ab."
\end{verbatim}

Ok, ich weiß, Sprache funktioniert nicht wirklich so. 
Aber wir stellen uns jetzt mal doof und tun so als ob.
 
Wir teilen den Satz in Worte auf und 
verteilen die Dauer gleichmäßig auf jedes Wort. Der Satz oben ist 9 Worte lang.

Also bekommt jedes Wort 1/9tel der Dauer assoziiert.

Damit haben wir eine einfache Tabelle mit Zeiten.

Jetzt können wir eine andere Aufgabe nehmen, also z.B. "Kaufe ein kleines Buch".

Wir kennen die zugeordneten Dauern zu "Kaufe", "ein" und "Buch". Wir 
werden etwas mit dem unbekannten Wort machen müssen. Es ignorieren oder 10\%
auf die Summe der bekannten Dauern aufschlagen. Irgendsowas. Und voilà, 
schon haben wir eine Zeitschätzung.

Lustige Sache:

Es ist tatsächlich in diesem Fall nicht ganz undenkbar, dass, in der Abwesenheit
weiterer Informationen, die zweite Aufgabe etwa halb so viel Zeit brauchen
könnte, wie die Aufgabe, von der wir "gelernt" haben.

Aber das ist nur in diesem Fall so. 

Egal. Wir haben eine Zeitschätzung und niemand wurde verletzt!

\subsection{Den Algorithmus über Powershell verwenden}

Der Algorithmus ist Teil von "EstimatePS". Das ist ein Powershell-Modul, 
welches auch über die Powershell-Gallery verfügbar ist.
Du kannst Dir gerne Source/experiments/duration-per-word/experiment1.ps1 als 
Referenz ansehen, wie es verwendet wird.

Am Ende ist es so einfach:

\begin{verbatim}
$inSekunden = Get-DPWEstimate -Model $model -DurationInSecondsFor "Kaufe ein Buch und lege es im Regal ab." -ProbabilityInPercent 95
\end{verbatim}

